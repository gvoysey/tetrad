\documentclass[10pt]{article}
\usepackage{pdfpages}
\usepackage{times}
\usepackage[margin = 1.0in]{geometry}
\usepackage[utf8]{inputenc}

\begin{document}
% frontmatter
\title{A proposal towards proposed research in computational biology}
\date{13 December 2017}
\author{Graham Voysey (gvoysey@bu.edu)\\
	    Kat Elkind (kelkind@bu.edu)\\
	    Rachel Petherbridge (rpether@bu.edu)\\
	    Kestutis Subacius (kestas@bu.edu)}
\maketitle
% main text
\section{Abstract}
% A 300-500 word summary of the problem statement and its challenge, the state of the art in the field, your approach, your results (this is the most important part, of course), and the broader impact of your contributions.
\section{Background}
% A brief summary (1000 words max) of the problem, the current state of the field and existing literature, situating your work in the context of the more general area of research, and clearly stating how it differs from previous approaches, or from your previous work and other undertakings.
\section{Results and Discussion}
% While some conferences/journals ask you to separate your results and their discussion, we prefer a single section describing for each aspect of your project, a summary of the challenge undertaken, a detailed description of the method used, and a clear description of the results obtained. With these descriptions, you can intersperse discussion on the rationale of the methods used, and also comment on the interpretation of the results you found. Be precise about your methods, and describe clearly any existing tools you used, and any new methods you developed (and their availability, if applicable). Be critical about your own results and findings, and think about alternative interpretations of your findings. Think about alternative approaches you would have considered if time permitted, or if you were to start anew.

\section{Future Goals}
% Because this is a class project, we want you to think hard not only about what you have accomplished in the period of the term, but also where you want to take this project going forward. This is a great place to describe your plans for publication of this work, what else you would like to include before it is ready for prime-time, whether it’s a master’s thesis, a Nature paper, or your next book. We like to see projects flourish even after the class ends, and great new directions are testament of great projects.

\section{Project Summary of Progress}
\subsection{Comparison with original and revised proposals}
% You also must comment on how the final report compares to what you proposed in the original and revised proposals, in one or two paragraphs. Did you address all reviewer’s concerns and how, did you manage to accomplish all your stated goals, which ones failed and why, which were added and why, was your proposal overly ambitious or vague or short. Explain the differences, what shaped the project away from the original aims, and could you have foreseen that earlier, or was it reflected in the reviewer comments.
\subsection{Commentary on experience}
% We want you to also go back and think critically about your project, and what advice you would give to yourself if you were to start over. Were the datasets well-suited for the approach, did you manage your time well between doing and writing, or between different aims of your project, what aspect of your project proved the most challenging / the most rewarding, what part of the project did you enjoy the most / the least.
\subsection{Commentary on Peer Review}
% One paragraph. How did the review process help shape your proposal? Did you find the feedback from other student’s useful/consistent/constructive? Having completed your project and looking back at the comments you received on your proposal, do you feel that the reviews were useful? Did they help identify and avoid potential pitfalls or problems? Or did they perhaps require additional work which in the end proved misleading?
\subsection{Division of Labor}
% One paragraph. For people working in groups, describe the contributions of each of the authors. This is standard in most scientific journals nowadays, and you should be completely honest about it, and as precise as possible. You can also comment on the good and the bad aspects of the collaboration, and what advice you would give yourself were you to start over.


\end{document}